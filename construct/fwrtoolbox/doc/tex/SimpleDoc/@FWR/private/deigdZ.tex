\begin{command}[@FWR/deigdZ]{deigdZ}
	\desc{Purpose}
Compute $M_1^\top \dd{\lambda}{A} M_2^\top$.
This function used to compute the pole sensitivity (open-loop and closed-loop)
	\desc{Syntax}
\matlab{[dlambda\_dZ, dlk\_dZ] = dleigdZ( A, M1, M2, Z, moduli)}
	\desc{Parameters}
		\begin{tabular}{l@{\ :\ }p{9cm}}
\matlab{dlambda\_dZ} &  the pole sensitivity matrix                                                                        \\
\matlab{dlk\_dZ} &  pole sensitivity matrices for each pole                                                                \\
\matlab{A} &  matrix from whom the eigenvalues are taken                                                                  \\
\matlab{M1,M2} &  such that $\dd{\lambda}{Z} = M1^\top \dd{\lambda}{A} M2^\top$                                           \\
\matlab{moduli } &  1 (default value) : compute $\dd{\abs{\lambda}}{Z}$ (the sensitivity of the moduli of the eigenvalues)\\
\matlab{} &  0 : compute $\dd{\lambda}{Z}$ (without the moduli)                                                           \\
		\end{tabular}
	\desc{Description}
\begin{center}\I{Internal function}\end{center}
This function computes
\begin{equation}
M_1^\top \dd{\lambda_k}{A} M_2^\top
\end{equation}
where the $\lambda_k$ are the eigenvalues of $A$.\\
This is done by the followin lemma\cite{Wu01}:
\begin{lemma}
Let $M\in\Rbb{n}{n}$ be diagonalisable. Let $\pa{\lambda_{k}}_{1 \leq k \leq n}$ be its eigenvalues, and $\pa{x_{k}}_{1 \leq k \leq n}$ the corresponding right eigenvectors. Denote $M_{x} \triangleq \begin{pmatrix}x_{1}, x_{2}, \hdots, x_{n}\end{pmatrix}$ and $M_{y} = \begin{pmatrix}y_{1}, y_{2}, \hdots, y_{n}\end{pmatrix} \triangleq M_{x}^{-H}$. Then
\begin{equation}\label{eq:dlambda}
\dd{\lambda_{k}}{M} = y^\ast_{k}x_{k}^\top \hspace{3mm} \forall k=1,\hdots,n
\end{equation}
and
\begin{equation}\label{eq:dmodulilambda}
\dd{\abs{\lambda_{k}}}{M} = \frac{1}{\abs{\lambda_{k}}}Re\pa{\lambda_{k}^\ast\dd{\lambda_{k}}{M}}
\end{equation}
where $\cdot^\ast$ denotes the conjugate operation, $Re(\cdot)$ the real part and $\cdot^H$ the transpose conjugate operator.
\end{lemma}
\desc{See also}
\funcName[@FWR/MsensPole]{MsensPole}, \funcName[@FWR/MsensPolecl]{MsensPole\_cl}, \funcName[@FWR/Mstability]{Mstability}
\end{command}


