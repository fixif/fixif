\begin{command}[@FWR/FWRmat2LaTeX]{FWRmat2LaTeX}
	\desc{Purpose}
Display a matrix ($Z$ or a sensitivity matrix) in \LaTeX (with \matlab{pmat}
package). The non trivial parameters (according to $W_Z$) are in bold.
	\desc{Syntax}
\matlab{S = FWRmat2LaTeX( R)               }\\
\matlab{S = FWRmat2LaTeX( R, M, format,tol)}
	\desc{Parameters}
		\begin{tabular}{l@{\ :\ }p{9cm}}
\matlab{S} &  string result (to be pasted in \LaTeX source) \\
\matlab{R} &  FWR object                                    \\
\matlab{M} &  matrix to print in \LaTeX format              \\
\matlab{} &  if $M$ is omitted (or empty), $Z$ is printed   \\
\matlab{} &  $M$ must have the same size as $Z$             \\
\matlab{format} &  print format                             \\
\matlab{} &  default value : \matlab{'\%0.5g'}                        \\
\matlab{} &  \matlab{'\%.3e'} for short e format, etc...              \\
\matlab{tol} &  tolerance to find trivial parameter (1,-1,0)\\
\matlab{} & default value: 1e-10                            \\
		\end{tabular}
	\desc{Description}
Display a matrix ($Z$ or a given matrix \matlab{M}, for example a sensitivity matrix) of a FWR object in a special \LaTeX format
(with \texttt{pmat} package): the coefficients with $W_{Z(i,j)}$ non null are in bold.
	\desc{Example}
The command \matlab{FWRmat2latex(R)}, where $R$ is a FWR object,
returns:
\begin{lstlisting}
\begin{pmat}({|...|})
\cr\-
&\mathbf{3.7673} & \mathbf{-1.8552} & \mathbf{1.0013} & \mathbf{-0.91839} & \mathbf{2} \cr
&\mathbf{4} & 0 & 0 & 0 & 0 \cr
&0 & \mathbf{2} & 0 & 0 & 0 \cr
&0 & 0 & \mathbf{0.5} & 0 & 0 \cr\-
&\mathbf{0.90722} & \mathbf{-0.56715} & \mathbf{0.24114} & \mathbf{-0.16096} & \mathbf{0.48163} \cr
\end{pmat}
\end{lstlisting}
When compiled, this \LaTeX code produces
$$\begin{pmat}({|...|})
\cr\-
&\mathbf{3.7673} & \mathbf{-1.8552} & \mathbf{1.0013} & \mathbf{-0.91839} & \mathbf{2} \cr
&\mathbf{4} & 0 & 0 & 0 & 0 \cr
&0 & \mathbf{2} & 0 & 0 & 0 \cr
&0 & 0 & \mathbf{0.5} & 0 & 0 \cr\-
&\mathbf{0.90722} & \mathbf{-0.56715} & \mathbf{0.24114} & \mathbf{-0.16096} & \mathbf{0.48163} \cr
\end{pmat}$$
\end{command}


