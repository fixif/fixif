\begin{command}[@FWR/l2scaling]{l2scaling}
	\desc{Purpose}
Perform a $L_2$-scaling on the FWR
	\desc{Syntax}
\matlab{R = l2scaling(R, Wcii)}\\
\matlab{[U,Y,W] = l2scaling(R)}
	\desc{Parameters}
		\begin{tabular}{l@{\ :\ }p{9cm}}
\matlab{R} &  FWR object                                                                 \\
\matlab{U,Y,W } &  transformation matrices applied on R                                  \\
\matlab{Wcii } &  vector (size $(1,l+n)$) of controllability gramians desired            \\
\matlab{} &  if \matlab{Wcii} is omitted, strict $L_2$-scaling is applied ( \matlab{Wcii=ones(1,n+l)} )\\
		\end{tabular}
	\desc{Description}
Perform a $L_2$-scaling.\\
The scaling forces the transfer functions from the inputs to the states and the intermediate variables to have a unitary $L_2$-norm. Theses norms are given by the diagonal terms of $W_c$ and $J^{-1}\pa{NN^\top+MW_cM^\top}J^{-\top}$.\\
The $L_2$-scaling is a $\mt{U}\mt{Y}\mt{W}$-transformation where $\mt{U}$ and $\mt{W}$ are diagonal with:
\begin{eqnarray}
\pa{\mt{U}}_{ii} &=& \sqrt{\pa{W_c}_{ii}} \\
\pa{\mt{W}}_{ii} &=& \sqrt{\pa{J^{-1}\pa{NN^\top+MW_cM^\top}J^{-\top}}_{ii}}
\end{eqnarray}
It is also possible to assign some particular values for the diagonal terms of the two gramians.
\desc{See also}
\funcName[@FWR/relaxedl2scaling]{relaxedl2scaling}
	\desc{References}
\cite{Feng09a} Y.�Feng, P.�Chevrel, and T.�Hilaire. A practival strategy of an efficient and sparse FWL implementation of LTI filters. In submitted to ECC'09, 2009.\\
%\cite{Hila09a}	T.�Hilaire. Low parametric sensitivity realizations with relaxed l2-dynamic-range-scaling constraints. submitted to IEEE Trans. on Circuits \& Systems II, 2009.
\end{command}


