\begin{command}[@FWR/realize]{realize}
	\desc{Purpose}
Numerically compute the outputs, states and intermediate variables with a given input U.
Floating-point is used for the computations.
	\desc{Syntax}
\matlab{[Y, X, T] = realize( R, U, X0)}
	\desc{Parameters}
		\begin{tabular}{l@{\ :\ }p{9cm}}
\matlab{Y} &  outputs                    \\
\matlab{X} &  states                     \\
\matlab{T} &  intermediate variables     \\
\matlab{R} &  FWR object                 \\
\matlab{U} &  inputs                     \\
\matlab{X0} &  initial states (default=0)\\
		\end{tabular}
	\desc{Description}
This function could be useful to evaluate the magnitude values of the intermediate variables and the states.\\
It could also be useful to compare two different realizations, for example a realization and its quantized one.\\
It is important to notice that the computations are done in floating-point (the fixed-point implementation is not considered here).
\end{command}


