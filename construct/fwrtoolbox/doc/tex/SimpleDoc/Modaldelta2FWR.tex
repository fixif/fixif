\begin{command}[Modaldelta2FWR]{Modaldelta2FWR}
	\desc{Purpose}
Transform a $\delta$-based modal realization into a FWR object
, and this realization can be relaxed $L_{2}$-scaled or not.
	\desc{Syntax}
\matlab{R = Modaldelta2FWR( SYS, Delta, isDeltaExact )           }\\
\matlab{R = Modaldelta2FWR( Aq, Bq, Cq, Dq, Delta, isDeltaExact )}
	\desc{Parameters}
		\begin{tabular}{l@{\ :\ }p{9cm}}
\matlab{R} &  FWR object                                                                                                          \\
\matlab{SYS } &  \matlab{ss} object                                                                                                      \\
\matlab{Aq,Bq,Cq,Dq } &  State-space ($q$-operator) matrices                                                                      \\
\matlab{Delta } &  Vector of $\Delta_i$. If they are not given, a relaxed $L_2$-scaling is performed ($\Delta_k$ then are induced)\\
\matlab{isDeltaExact } &  1 if the vector of $\Delta_i$ is exactly implemented                                                    \\
\matlab{} &  0 (default value) else                                                                                               \\
		\end{tabular}
	\desc{Description}
Let consider the following transfer function given and its related modal realization $(\Lambda, B, C, D)$:
\begin{equation}\label{eq:Modaldelta2FWR:modal_transfer_function}
\begin{split}
H(z) & =D+C(zI-\Lambda)^{-1}B \\
& =D+\sum_{i=1}^{n}\frac{c_{i}b{i}}{1-\lambda_{i}z^{-1}}
\end{split}
\end{equation}
with $\lambda_{i}\neq\lambda_{j}$ for all $i\neq j$ so that $\Lambda$ may be chosen as a diagonal matrix.\\
The modal representation is not unique since $B$ and $C$ may be scaled and the diagonal elements of $\Lambda$ may be permuted in different ways so as to to produce the same transfer function. One invariant however is that is decouples the dynamic modes $\lambda_{i}$ and is closely related to partial-fraction expansion of $H(z)$. Rather to diagonalize the $A$-matrix, it is preferred in the sequel to combine the complex-conjugate pole-pairs to form a real ``block-diagonal'' section in which $\Lambda$ has two-by-two real matrices along its diagonal as follows:
\begin{equation}\label{eq:Modaldelta2FWR:modal_representation}
\Lambda=\begin{pmatrix}
\alpha_{1} & \beta_{1} &  &  &  &  &  \\
\beta_{2} & \alpha_{2} &  &  &  &  &  \\
&  & \alpha_{3} & \beta_{3} &  &  &  \\
&  & \beta_{4} & \alpha_{4} &  &  &  \\
&  &  &  & \ddots &  &  \\
&  &  &  &  & \alpha_{n-1} & \beta_{n-1} \\
&  &  &  &  & \beta_{n} & \alpha_{n} \\
\end{pmatrix}
\end{equation}
where $\alpha_{i}$ and $\beta_{i}$ are linked to the real part and the
imaginary part of the $i^{th}$ pole, respectively. If the $i^{th}$ pole is real, then $\beta_{i}=0$; if the $i^{th}$ and $(i+1)^{th}$ poles are complex-conjugate, then $\alpha_{i}=\alpha_{i+1}$ and $\beta_{i}=-\beta_{i+1}=Im(\lambda_{i})$.\\
The system considered is described by the equations
\begin{equation}
\left\lbrace\begin{array}{rcl}
\delta[X(k)] &=& A_\delta X(k) + B_\delta U(k) \\
Y(k) &=& C_\delta X(k) + D_\delta U(k)
\end{array}\right.
\end{equation}
where the $\delta$-operator is defined by
\begin{equation}
\delta_i \triangleq \frac{q-1}{\Delta_i}
\end{equation}
and $\Delta_i$ is a strictly positive constant.\\
The (finite precision) equivalent system, in the implicit state-space formalism, is given by
\begin{equation}\label{eq:modaldelta2FWR:implicit_delta2}
\begin{pmatrix}
I_{n} & 0 & 0\\
-\Delta & I_{n} & 0\\
0 & 0 & I_{p}
\end{pmatrix}
\begin{pmatrix}
T(k+1)\\
X(k+1)\\
Y(k)
\end{pmatrix}
=
\begin{pmatrix}
0 & A_\delta & B_\delta\\
0 & I_{n} & 0\\
0 & C_\delta & D_\delta\\
\end{pmatrix}
\begin{pmatrix}
T(k)\\
X(k)\\
U(k)
\end{pmatrix}
\end{equation}
%		or
%		\begin{equation}
%			Z =
%			\begin{pmatrix}
%				-I_n & A_\delta & B_\delta \\
%				\Delta I_{n} & I_n & 0 \\
%				0 & C_\delta & D_\delta
%			\end{pmatrix}
%		\end{equation}
where
\begin{equation*}
\Delta=\hbox{diag}(\Delta_{1}~\cdots~\Delta_{n})
\end{equation*}
If the system is given in classical state-space (\matlab{ss} object), the equivalent $\delta$-realization is obtained with :
\begin{equation}
A_\delta = \Delta^{-1}(\Lambda-I_n), \quad B_\delta = \Delta^{-1}B,
\quad C_\delta=C, \quad D_\delta = D
\end{equation}
The \matlab{isDeltaExact} parameter determines $W_K$.
\desc{See also}
\funcName[Modalrho2FWR]{Modalrho2FWR}
	\desc{References}
\cite{Midd90a} R.~Middleton and G.~Goodwin, Digital Control and Estimation, a unified approach, Prentice-Hall International Editions, 1990.\\
\cite{Feng09a} Y.�Feng, P.�Chevrel, and T.�Hilaire. A practival strategy of an efficient and sparse fwl implementation of lti filters. In submitted to ECC'09, 2009.\\
\end{command}


