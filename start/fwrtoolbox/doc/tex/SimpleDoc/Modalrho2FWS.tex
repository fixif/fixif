\begin{command}[Modalrho2FWS]{Modalrho2FWS}
	\desc{Purpose}
Transform a $\rho$-modal realization into a FWS object, where the parameter $\Delta$ is reserved for relaxed $L_{2}$-scaling
	\desc{Syntax}
\matlab{S = Modalrho2FWS( SYS, Gamma,  Gamma, isGammaExact )    }\\
\matlab{S = Modalrho2FWS( Aq, Bq, Cq, Dq,  Gamma, isGammaExact )}
	\desc{Parameters}
		\begin{tabular}{l@{\ :\ }p{9cm}}
\matlab{S} &  FWS object                                                                                       \\
\matlab{SYS } &  Initial classical $q$-state-space system to be converted                                      \\
\matlab{Aq,Bq,Cq,Dq } &  State-space ($q$-operator) matrices                                                   \\
\matlab{Gamma } &  Vector of $gamma_i$ parameters                                                              \\
\matlab{isGammaExact } &  1 (default value) if we consider that the vector of $\gamma_i$ is exactly implemented\\
\matlab{} &  0 else                                                                                            \\
		\end{tabular}
	\desc{Description}
% The modal representation applied here is the same as that used in
\funcName[@FWR/Modaldelta2FWR]{Modaldelta2FWR} (see the details therein),
while $\Delta$ is reserved for relaxed $L_2$-scaling.
Define the following series of $1^{st}$ polynomial operators, named $\rho$-operators:
\begin{equation}\label{eq:Modalrho2FWS:rho_operator}
\rho_{i}=\frac{q-\gamma_{i}}{\Delta_{i}},\quad\forall i=1,2,\cdots,n
\end{equation}
with $\alpha_{i}$ and $\Delta_{i}>0$ are two sets of constants to determine. The particular choice $\alpha_{i}=0$ and $\Delta_{i}=1$ (resp. $\alpha_{i}=1$) leads to the shift operator (resp. the $\delta$-operator). The specialized implicit form related to the $\rho$-operator has the particular structure:
\begin{equation}\label{eq:Modalrho2FWS:rho_implicit}
\begin{pmatrix}
I & 0 & 0 \\
-\Delta & I & 0 \\
0 & 0 & I \\
\end{pmatrix}
\begin{pmatrix}
T_{k+1} \\
X_{k+1} \\
Y_{k}
\end{pmatrix}
=
\begin{pmatrix}
0 & A_{\rho} & B_{\rho} \\
0 & \gamma & 0 \\
0 & C_{\rho} & D_{\rho} \\
\end{pmatrix}
\begin{pmatrix}
T_{k} \\
X_{k} \\
U_{k}
\end{pmatrix}
\end{equation}
The condition of keeping equivalece is given as below:
\begin{eqnarray}
\label{eq:Modalrho2FWS:rho_state}A_{\rho}=\Delta^{-1}(\Lambda-1), B_{\rho}=\Delta^{-1}B, C_{\rho}=C~\hbox{and}~D_{\rho}=D\\
\label{eq:Modalrho2FWS:Delta_gamma}\Delta=\hbox{diag}(\Delta_{1}~\cdots~\Delta_{n}),\quad\gamma=\hbox{diag}(\gamma_{1}~\cdots~\gamma_{n})
\end{eqnarray}
The \matlab{isGammaExact} parameters determines $W_P$.
All equivalent $\delta$-state-space realizations (with same size) can be obtained by modification of operator which is achieved by choosing different $\gamma_{i}$. So there are only one parameter in the structuration's definition, namely \matlab{$\gamma$}. \\
\desc{See also}
\funcName[Modalrho2FWR]{Modalrho2FWR}
	\desc{References}
\cite{Feng09a} Y.�Feng, P.�Chevrel, and T.�Hilaire. A practival strategy of an efficient and sparse FWL implementation of LTI filters. Submitted to ECC'09, 2009.\\
\end{command}


